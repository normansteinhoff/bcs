\section{Chrome Canary und Gemini Nano}

\subsection{Was ist Crome Canary}

Chrome Canary ist die experimentelle Version von Googles Chrome Browser und bietet die Möglichkeit, die neuesten Funktionen zu testen. Canary kann als eine Art Labor betrachtet werden, in dem neue Ideen und Features ausprobiert werden, bevor sie in die endgültige Version von Chrome aufgenommen werden. Diese Vorabversion wird täglich aktualisiert, wodurch die neuesten Entwicklungen stets verfolgt werden können. Da es sich jedoch um eine experimentelle Version handelt, kann es vorkommen, dass Canary instabil ist oder Fehler aufweist. \cite{chrome-canary}

\subsection{Gemini Nano}

Gemini Nano ist ein von Google entwickeltes Sprachmodell \cite{gemini-nano}. Der Hauptzweck von Gemini Nano ist es, KI Funktionen clientseitig ausführen zu können, ohne dafür jedesmal ein Sprachmodell herunterladen zu müssen. Dazu wurde Gemini Nano direkt in Chrome (Canary) integriert. \cite{gemini-nano-build-in-ai}

Aktuell befinden sich Gemini Nano und die dazugehörige API in einer Testphase. In diesem Zustand können und haben sich \footnote{Wie wir mehrfach während der Entwicklung unseres Plugins feststellen mussten} deren Funktionen geändert. \cite{gemini-nano-build-in-ai}

Die Gemini API ist in verschiedene Task API's aufgeteilt. Aktuell sind dies \cite{gemini-nano-apis}:
\begin{itemize}
    \item LanguageDetection
    \item Translation
    \item Summary
    \item Writer + Rewriter
    \item PromptApi \footnote{Unser Plugin stützt sich letzlich auf die PromptApi, auch wenn wir zwischendurch die anderen API's ausprobiert haben.}
\end{itemize} 

\subsection{Dokumentation von Gemini Nano}

Um die oben genannten API's nutzen zu können sollte man dem Origin Trial beiteten. Dies ging über \cite{gemini-nano-origin-trial}. Hier erhiehlten wir die jeweils neuesten Informationen via Email. Dies war notwendig, da die eigentliche Dokumentation \cite{old-doku-language-detection-api, old-doku-translation-api,old-doku-summarization-api,old-doku-writer-api,old-doku-prompt-api} seit längerer Zeit nicht mehr aktuell ist (Stand 04.02.2025). Inzwischen gibt es zwar über \cite{gemini-nano-build-in-ai} eine neue öffentlich zugängliche Dokumentation, allerdings ist diese ebenfalls nicht vollständig.

\subsection{Notwendige Voraussetzungen}

Um Gemini Nano in Chrome Canary nutzen zu können müssen mehrere Flags gesetzt werden. Dazu gibt man in der Adresszeile \textbf{chrome://flags} ein. Dort müssen dann folgende Flags gesetzt werden:
\begin{itemize}
    \item Enables optimization guide on device --> \emph{Enabled BypassPerfRequirement}
    \item Text Safety Classifier \footnote{Wenn der Text Safety Classifier nicht ausgeschaltet ist, weigert sich Gemini Nano Texte zu verarbeiten, die nicht in Englsich sind.} --> \emph{Disabled}
    \item Language detection web platform API --> \emph{Enabled}
    \item Experimental translation API --> \emph{Enabled without language pack limit}
    \item Summarization API for Gemini Nano --> \emph{Enabled}
    \item Writer API for Gemini Nano --> \emph{Enabled}
    \item Rewriter API for Gemini Nano --> \emph{Enabled}
    \item Prompt API for Gemini Nano --> \emph{Enabled}
\end{itemize}

Desweiteren sollte man danach in der Adresszeile \textbf{chrome://components/} eingeben und dann \emph{Optimization Guide On Device Model} suchen und aktualisieren.