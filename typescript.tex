\section{Typescript}

\subsection{Was ist TypeScript}

Typescript (TS) ist eine von Microsoft entwickelte Erweiterung von JavaScript (JS). \cite{typescript}

TS wird zu JS convertiert, wodurch der geschriebene Code in nahezu jedem Browser läuft. \cite{typescript}

TS erweitert JS durch ein Typsystem in dem unter anderem Interfaces und Klassen beschrieben werden können. Dadurch reduzieren sich die möglichen Fehlerquellen enorm, da der TS-Compiler einfache Typ-Überprüfungen vollziehen kann. Desweiteren ist dadurch auch eine bessere Integration mit IDE's wie zum Beispiel VSCode gegeben. \cite{typescript}

TS ist dabei so entworfen worden, dass jeder gültige JS-Code auch gültiger TS-Code ist. Es ist also möglich ein bestehendes JS-Projekt ohne weiteres auf TS umzustellen. Anschließend kann man, wenn man möchte, den Code vorhandenen Code zu TS convertieren. \cite{typescript} \cite{ts-doku} Dabei hilft ein Feature von TS. Die so genannte tsconfig.json Datei enthält Einstellungen für den TS-Compiler um beispielsweise erweitertes ErrorChecking zu ermöglichen. \cite{tsconfig}

\subsection{Wie haben wir Typescript genutzt?}

Aufgrund der vielen Vorteile von TS haben wir unsere Chrome Extension direkt und vollständig (100 \%) in TS geschrieben.

Außerdem haben wir die tsconfig.json datei so eingestellt, dass der TS-Compiler maximal streng vorgeht. Dies bedeutet, dass wir so viele Warnungen und Vorschläge vom Compiler bekommen möchten, wie irgend möglich. Außerdem wurde eingestellt, dass jede Warnung als Fehler zu interpretieren ist und zum Abbruch führt. Die Referenz für die tsconfig.json findet sich unter \cite{tsconfig}.

% \todo{shellscript}